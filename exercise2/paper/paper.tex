%x%%%%%%%%%%%%%%%%%%%%%%%%%%%%%%%%%%%%%%%%
% Short Sectioned Assignment
% LaTeX Template
% Version 1.0 (5/5/12)
%
% This template has been downloaded from:
% http://www.LaTeXTemplates.com
%
% Original author:
% Frits Wenneker (http://www.howtotex.com)
%
% License:
% CC BY-NC-SA 3.0 (http://creativecommons.org/licenses/by-nc-sa/3.0/)
%
%%%%%%%%%%%%%%%%%%%%%%%%%%%%%%%%%%%%%%%%%

%----------------------------------------------------------------------------------------
%   PACKAGES AND OTHER DOCUMENT CONFIGURATIONS
%----------------------------------------------------------------------------------------

\documentclass[paper=a4, fontsize=11pt]{scrartcl} % A4 paper and 11pt font size

\usepackage{color}
\usepackage{float}
\usepackage{placeins}
\usepackage{natbib}
% philipp: added for code listings
\usepackage{listings}
\lstset{
    captionpos=b
}
\newcommand{\Hilight}{\makebox[0pt][l]{\color{red}\rule[-4pt]{0.65\linewidth}{14pt}}}
\usepackage{booktabs}
\usepackage{multirow}
\usepackage{morefloats}
\bibliographystyle{plain}

\usepackage[T1]{fontenc} % Use 8-bit encoding that has 256 glyphs
\usepackage{fourier} % Use the Adobe Utopia font for the document - comment this line to return to the LaTeX default
\usepackage[english]{babel} % English language/hyphenation
\usepackage{amsmath,amsfonts,amsthm} % Math packages

\usepackage{lipsum} % Used for inserting dummy 'Lorem ipsum' text into the template
\usepackage{todonotes}

\usepackage{sectsty} % Allows customizing section commands
\allsectionsfont{\centering \normalfont\scshape} % Make all sections centered, the default font and small caps

\usepackage{fancyhdr} % Custom headers and footers
\pagestyle{fancyplain} % Makes all pages in the document conform to the custom headers and footers
\fancyhead{} % No page header - if you want one, create it in the same way as the footers below
\fancyfoot[L]{} % Empty left footer
\fancyfoot[C]{} % Empty center footer
\fancyfoot[R]{\thepage} % Page numbering for right footer
% \renewcommand{\thesection}{Data Set \arabic{section}}
\renewcommand{\headrulewidth}{0pt} % Remove header underlines
\renewcommand{\footrulewidth}{0pt} % Remove footer underlines
\setlength{\headheight}{13.6pt} % Customize the height of the header

\numberwithin{equation}{section} % Number equations within sections (i.e. 1.1, 1.2, 2.1, 2.2 instead of 1, 2, 3, 4)
\numberwithin{figure}{section} % Number figures within sections (i.e. 1.1, 1.2, 2.1, 2.2 instead of 1, 2, 3, 4)
\numberwithin{table}{section} % Number tables within sections (i.e. 1.1, 1.2, 2.1, 2.2 instead of 1, 2, 3, 4)

\setlength\parindent{0pt} % Removes all indentation from paragraphs - comment this line for an assignment with lots of text

%----------------------------------------------------------------------------------------
%   TITLE SECTION
%----------------------------------------------------------------------------------------

\newcommand{\horrule}[1]{\rule{\linewidth}{#1}} % Create horizontal rule command with 1 argument of height

\title{ 
\normalfont \normalsize 
\textsc{Vienna University of Technology} \\ [25pt] % Your university, school and/or department name(s)
\horrule{0.5pt} \\[0.4cm] % Thin top horizontal rule
\huge Experiments in Machine Learning 2 \\ % The assignment title
\horrule{2pt} \\[0.5cm] % Thick bottom horizontal rule
}

\author{Benjamin Kiesl \and Philipp Steinwender \and Robert Sch\"{a}fer} % Your name

\date{\normalsize\today} % Today's date or a custom date

\begin{document}

\maketitle % Print the title

%----------------------------------------------------------------------------------------
%   PROBLEM 1
%----------------------------------------------------------------------------------------

\tableofcontents

\section{TODO}

\begin{itemize}
\item Structure for paper
\item Implementation of cutsom bn-searcher
\item documentation of cutsom bn-searcher
\item comparison and interpretation of the results
\item description of the used R and Weka features and commands
\item description of the setup/infrastructure (weka, custom searcher, R, runtime, comparison, data set, ...)
\item R script that runs the custom searcher
\item R script that runs a default searcher
\item find and describe a custom search algorithm we want to uses
\end{itemize}

\section{Preface}

\paragraph{}This assignment is about extending a Bayes Network Classifier by adding a custom search algorithm.

\section{Setup}

Used stuff: R. RWeka -- an interface in R for Weka. implement a searcher by extending a java class from the weka library/framework. running a R script that starts the weka BayesNet-classifier with our custom searcher.

\subsection{RWeka}

RWeka is a interface in R that wraps the Weka framework and has a really good documentation.

Some good stuff to know:

\texttt{WOW("at.ac.tuwien.machine\_learning.UltraSearcher")} WOW is a function in R that reads and prints all possible options of a weka class. This can be done for classifiers, filters, searchers and all other types in weka.

\subsection{Weka Javacode}

\paragraph{initStructure()} of class BayesNet initializes the Bayesnetwork. It creates an adjacency matrix where each attribute has an array of parents. These arrays are empty after initialization.

\paragraph{} The default SearchAlgorithm.class initializes the network like a Naive Bayes network. This shows how we can use the adjacency matrix of BayesNet:

\begin{lstlisting}
...
// initialize parent sets to have arrow from classifier node to
// each of the other nodes
for (int iAttribute = 0; iAttribute < instances.numAttributes(); iAttribute++) {
    if (iAttribute != iClass) {
        bayesNet.getParentSet(iAttribute).addParent(iClass, instances);
    }
}
...
\end{lstlisting}

\paragraph{} Where \texttt{BayesNet} is the classifier that holds the network datastructure and \texttt{getParentSet(n)} returns the \texttt{nth} row in the adjacency matrix.









\bibliography{references}


\end{document}
%------------------------------------------------


