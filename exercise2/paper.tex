%x%%%%%%%%%%%%%%%%%%%%%%%%%%%%%%%%%%%%%%%%
% Short Sectioned Assignment
% LaTeX Template
% Version 1.0 (5/5/12)
%
% This template has been downloaded from:
% http://www.LaTeXTemplates.com
%
% Original author:
% Frits Wenneker (http://www.howtotex.com)
%
% License:
% CC BY-NC-SA 3.0 (http://creativecommons.org/licenses/by-nc-sa/3.0/)
%
%%%%%%%%%%%%%%%%%%%%%%%%%%%%%%%%%%%%%%%%%

%----------------------------------------------------------------------------------------
%   PACKAGES AND OTHER DOCUMENT CONFIGURATIONS
%----------------------------------------------------------------------------------------

\documentclass[paper=a4, fontsize=11pt]{scrartcl} % A4 paper and 11pt font size

\usepackage{color}
\usepackage{float}
\usepackage{placeins}
\usepackage{natbib}
% philipp: added for code listings
\usepackage{listings}
\lstset{
    captionpos=b
}
\newcommand{\Hilight}{\makebox[0pt][l]{\color{red}\rule[-4pt]{0.65\linewidth}{14pt}}}
\usepackage{booktabs}
\usepackage{multirow}
\usepackage{morefloats}
\bibliographystyle{plain}

\usepackage[T1]{fontenc} % Use 8-bit encoding that has 256 glyphs
\usepackage{fourier} % Use the Adobe Utopia font for the document - comment this line to return to the LaTeX default
\usepackage[english]{babel} % English language/hyphenation
\usepackage{amsmath,amsfonts,amsthm} % Math packages

\usepackage{lipsum} % Used for inserting dummy 'Lorem ipsum' text into the template
\usepackage{todonotes}

\usepackage{sectsty} % Allows customizing section commands
\allsectionsfont{\centering \normalfont\scshape} % Make all sections centered, the default font and small caps

\usepackage{fancyhdr} % Custom headers and footers
\pagestyle{fancyplain} % Makes all pages in the document conform to the custom headers and footers
\fancyhead{} % No page header - if you want one, create it in the same way as the footers below
\fancyfoot[L]{} % Empty left footer
\fancyfoot[C]{} % Empty center footer
\fancyfoot[R]{\thepage} % Page numbering for right footer
% \renewcommand{\thesection}{Data Set \arabic{section}}
\renewcommand{\headrulewidth}{0pt} % Remove header underlines
\renewcommand{\footrulewidth}{0pt} % Remove footer underlines
\setlength{\headheight}{13.6pt} % Customize the height of the header

\numberwithin{equation}{section} % Number equations within sections (i.e. 1.1, 1.2, 2.1, 2.2 instead of 1, 2, 3, 4)
\numberwithin{figure}{section} % Number figures within sections (i.e. 1.1, 1.2, 2.1, 2.2 instead of 1, 2, 3, 4)
\numberwithin{table}{section} % Number tables within sections (i.e. 1.1, 1.2, 2.1, 2.2 instead of 1, 2, 3, 4)

\setlength\parindent{0pt} % Removes all indentation from paragraphs - comment this line for an assignment with lots of text

%----------------------------------------------------------------------------------------
%   TITLE SECTION
%----------------------------------------------------------------------------------------

\newcommand{\horrule}[1]{\rule{\linewidth}{#1}} % Create horizontal rule command with 1 argument of height

\title{ 
\normalfont \normalsize 
\textsc{Vienna University of Technology} \\ [25pt] % Your university, school and/or department name(s)
\horrule{0.5pt} \\[0.4cm] % Thin top horizontal rule
\huge Experiments in Machine Learning 2 \\ % The assignment title
\horrule{2pt} \\[0.5cm] % Thick bottom horizontal rule
}

\author{Benjamin Kiesl \and Philipp Steinwender \and Robert Sch\"{a}fer} % Your name

\date{\normalsize\today} % Today's date or a custom date

\begin{document}

\maketitle % Print the title

%----------------------------------------------------------------------------------------
%   PROBLEM 1
%----------------------------------------------------------------------------------------

\tableofcontents

\section{Preface}

\paragraph{}This assignment is about extending a Bayes Network Classifier by adding a custom search algorithm.

\section{Setup}

Used stuff: R. RWeka -- an interface in R for Weka. implement a searcher by extending a java class from the weka library/framework. running a R script that starts the weka BayesNet-classifier with our custom searcher.

\subsection{Preprocessing}

\paragraph{}Write about how we preprocessed the data.


\subsection{Classifier}

=There is a beautiful confusion matrix:

\begin{table*}[htb]\centering
    \begin{tabular}{ll|ll} 
\multicolumn{2}{c}{\phantom{bla}}  & \multicolumn{2}{c}{Classified as} \\  
                        & \phantom{aa}  &  0  &  1  \\ \cmidrule{2-4}
\multirow{2}{*}{Instances}     &  0  &  0  &  4  \\
                                   &  1  &  1  &  90 \\ 
\end{tabular}
\caption{Confusion matrix for Decision tree}
\end{table*}

And a listing:

\begin{lstlisting}
J48 unpruned tree
pericardial-effusion = 0
|   epss <= 13100: 1 (76.62/1.95)
|   epss > 13100
|   |   epss <= 15600: 0 (2.19/0.16)
|   |   epss > 15600: 1 (2.19/0.03)
pericardial-effusion = 1: 1 (14.0)
\end{lstlisting}

And look at this performance/accuracy table:

\begin{table*}[htb]\centering
    \begin{tabular*}{\columnwidth}{@{}lllllll@{}}
        \toprule 
               &  TP Rate & FP Rate & Precision & Recall & F-Measure &  Class  \\ \midrule
               &  0       & 0       & 0         & 0      & 0         &  0      \\     
               &  1       & 1       & 0.959     & 1      & 0.979     &  1      \\     
Weighted Avg.  &  0.959   & 0.959   & 0.919     & 0.959  & 0.939     &         \\ \bottomrule
    \end{tabular*}
\caption{Decision Tree -- best achievable results} 
\label{tab:echo:dec:best}
\end{table*}
\FloatBarrier

\subsubsection{Another Classifier}

Look what I can predict!
	


\bibliography{references}


\end{document}
%------------------------------------------------


