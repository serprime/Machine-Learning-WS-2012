%%%%%%%%%%%%%%%%%%%%%%%%%%%%%%%%%%%%%%%%%
% Short Sectioned Assignment
% LaTeX Template
% Version 1.0 (5/5/12)
%
% This template has been downloaded from:
% http://www.LaTeXTemplates.com
%
% Original author:
% Frits Wenneker (http://www.howtotex.com)
%
% License:
% CC BY-NC-SA 3.0 (http://creativecommons.org/licenses/by-nc-sa/3.0/)
%
%%%%%%%%%%%%%%%%%%%%%%%%%%%%%%%%%%%%%%%%%

%----------------------------------------------------------------------------------------
%   PACKAGES AND OTHER DOCUMENT CONFIGURATIONS
%----------------------------------------------------------------------------------------

\documentclass[paper=a4, fontsize=11pt]{scrartcl} % A4 paper and 11pt font size

% philipp: added for code listings
\usepackage{listings}

\usepackage{natbib}
\bibliographystyle{plain}

\usepackage[T1]{fontenc} % Use 8-bit encoding that has 256 glyphs
\usepackage{fourier} % Use the Adobe Utopia font for the document - comment this line to return to the LaTeX default
\usepackage[english]{babel} % English language/hyphenation
\usepackage{amsmath,amsfonts,amsthm} % Math packages

\usepackage{lipsum} % Used for inserting dummy 'Lorem ipsum' text into the template
\usepackage{todonotes}

\usepackage{sectsty} % Allows customizing section commands
\allsectionsfont{\centering \normalfont\scshape} % Make all sections centered, the default font and small caps

\usepackage{fancyhdr} % Custom headers and footers
\pagestyle{fancyplain} % Makes all pages in the document conform to the custom headers and footers
\fancyhead{} % No page header - if you want one, create it in the same way as the footers below
\fancyfoot[L]{} % Empty left footer
\fancyfoot[C]{} % Empty center footer
\fancyfoot[R]{\thepage} % Page numbering for right footer
% \renewcommand{\thesection}{Data Set \arabic{section}}
\renewcommand{\headrulewidth}{0pt} % Remove header underlines
\renewcommand{\footrulewidth}{0pt} % Remove footer underlines
\setlength{\headheight}{13.6pt} % Customize the height of the header

\numberwithin{equation}{section} % Number equations within sections (i.e. 1.1, 1.2, 2.1, 2.2 instead of 1, 2, 3, 4)
\numberwithin{figure}{section} % Number figures within sections (i.e. 1.1, 1.2, 2.1, 2.2 instead of 1, 2, 3, 4)
\numberwithin{table}{section} % Number tables within sections (i.e. 1.1, 1.2, 2.1, 2.2 instead of 1, 2, 3, 4)

\setlength\parindent{0pt} % Removes all indentation from paragraphs - comment this line for an assignment with lots of text

%----------------------------------------------------------------------------------------
%   TITLE SECTION
%----------------------------------------------------------------------------------------

\newcommand{\horrule}[1]{\rule{\linewidth}{#1}} % Create horizontal rule command with 1 argument of height

\title{ 
\normalfont \normalsize 
\textsc{Vienna University of Technology} \\ [25pt] % Your university, school and/or department name(s)
\horrule{0.5pt} \\[0.4cm] % Thin top horizontal rule
\huge Experiments in Machine Learning \\ % The assignment title
\horrule{2pt} \\[0.5cm] % Thick bottom horizontal rule
}

\author{Benjamin Kiesl \and Philipp Steinwender \and Robert Sch\"{a}fer} % Your name

\date{\normalsize\today} % Today's date or a custom date

\begin{document}

\maketitle % Print the title

%----------------------------------------------------------------------------------------
%   PROBLEM 1
%----------------------------------------------------------------------------------------

\section{Echocardiogram data}

The Echocardiogram data set consists of 132 entries with 13 attributes respectively and has 132 missing values in total. Originally, these entries referred to patients that suffered from a heart attack at some point in the past \cite{uci-repo}. 

\subsection{Preprocessing}
The name attribute of the patients was replaced by a dummy for all entries. For this reason, this attribute can be entirely removed. There is another peculiarity with the attribute ``alive-at-1'': Some of the patients survived less than 12 months even if they were still alive at the end of the survey. In this case these patients joined the survey later and weren't followed a whole year. If the attribute ``alive-at-1'' is set to 1, it means the person survived one year, otherwise the
person died or joined the survey later on. This attribute can be fully derived from the first two attributes, so it can be removed without interfering the results. Another unnecessary attribute is ``group'' which is just meaningless and can be ignored \cite{uci-repo}. Another derived attribute is ``mult'' and can be ignored as well.
\todo{Compare results with/without preprocessing}

%------------------------------------------------


\section{Adult data}

The Adult or Census Income data sets contains 48842 instances of information about persons and if they earn more than 50.000 a year. The data set has 14 attributes and missing values.

The instances origin from the 1994 Census Database and were extracted by Barry Becker. For having a clean data set, only records with age above 16 with minimium one working hour per week have been selected from the database.

\subsection{Preprocessing}

We can remove the final weight -- but we have to find out what it means !!.

We can also remove the $education-num$ attribute as it is only a numeric representation of the education attribute.

At first we try to discretize all numeric attributes with 10 bins of equal ranges.

\subsection{Decision Tree}

\subsection{Random Forrest}

\subsection{Support Vector Machine}

\subsection{Naive Bayes}

We use 66\% of the data set for training and the rest for evaluation.

For the first run, we use the default setting and get the following result:

\begin{verbatim}
Correctly Classified Instances        9104               82.2329 %
Incorrectly Classified Instances      1967               17.7671 %
Total Number of Instances            11071     

Precision   Recall  F-Measure  Class
 0.592       0.786   0.675      >50K
 0.927       0.834   0.878      <=50K
\end{verbatim}

Second run: we use the discretizer with 20 bin.

\begin{verbatim}
Correctly Classified Instances        9287               83.8858 %
Incorrectly Classified Instances      1784               16.1142 %
Total Number of Instances            11071     

=== Detailed Accuracy By Class ===

Precision   Recall  F-Measure  Class
 0.621       0.805   0.701      >50K
 0.934       0.849   0.89       <=50K
\end{verbatim}

As expected, the result is slightly better.

A better result should be possible if we let the dicretizer split the values in part of nearly equal size. The result:

\begin{verbatim}
Correctly Classified Instances        9179               82.9103 %
Incorrectly Classified Instances      1892               17.0897 %
Total Number of Instances            11071     

Precision   Recall  F-Measure  Class
0.602        0.805   0.689      >50K
0.933        0.836   0.882      <=50K
\end{verbatim}

It got worse. \todo{why does discretizer with equal bin size not give a better result?}

The Naive Bayes does not have a lot of settings. It use a built in supervised discretizer or a kernel estimator. But the result is always the same with around 80\% correctly classified instances in the test run.

\subsection{k-Nearest}


%------------------------------------------------


\subsection{Heading on level 2 (subsection)}

Lorem ipsum dolor sit amet, consectetuer adipiscing elit. 
\begin{align}
A = 
\begin{bmatrix}
A_{11} & A_{21} \\
A_{21} & A_{22}
\end{bmatrix}
\end{align}
Aenean commodo ligula eget dolor. Aenean massa. Cum sociis natoque penatibus et magnis dis parturient montes, nascetur ridiculus mus. Donec quam felis, ultricies nec, pellentesque eu, pretium quis, sem.

%------------------------------------------------

\subsubsection{Heading on level 3 (subsubsection)}

\lipsum[3] % Dummy text

\paragraph{Heading on level 4 (paragraph)}

\lipsum[6] % Dummy text

%----------------------------------------------------------------------------------------
%   PROBLEM 2
%----------------------------------------------------------------------------------------

\section{Lists}

%------------------------------------------------

\subsection{Example of list (3*itemize)}
\begin{itemize}
    \item First item in a list 
        \begin{itemize}
        \item First item in a list 
            \begin{itemize}
            \item First item in a list 
            \item Second item in a list 
            \end{itemize}
        \item Second item in a list 
        \end{itemize}
    \item Second item in a list 
\end{itemize}

%------------------------------------------------

\subsection{Example of list (enumerate)}
\begin{enumerate}
\item First item in a list 
\item Second item in a list 
\item Third item in a list
\end{enumerate}

%----------------------------------------------------------------------------------------



\bibliography{references}


\end{document}
